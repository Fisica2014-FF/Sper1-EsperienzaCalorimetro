\documentclass[10pt,a4paper]{article} % Prepara un documento con un font grande

\input{./preamboli_e_stili/pacchetti.tex}

\DeclareGraphicsExtensions{.pdf, .png, .jpg} % Se due immagini hanno lo stesso nome sceglile secondo l'ordine di filetype qui
\graphicspath{ {./img/} }					 % Path delle immagini 

\input{./preamboli_e_stili/titolo_Calorimetro.tex}
\input{./preamboli_e_stili/stili_float.tex}


%////////////////////////////////////////////////////////////////////////////////////////////////////////////////////////////
%////////////////////////////////////////////////////////////////////////////////////////////////////////////////////////////
% Fine dei dati iniziali per il latex: il documento finale inizierà da qui
\begin{document}

\maketitle % Produce il titolo a partire dai comandi \title, \author e \date

\tableofcontents % Prepara l'indice generale

% Le varie sezioni
%\section{Obiettivi}
\begin{abstract}
	Sima del calore specifico di un corpo.
Riconoscimento del materiale di un corpo attraverso il confronto del suo calore specifico con le tabelle fornite.


\end{abstract}

\section{Descrizione dell'apparato strumentale}
	


\section{Metodologia di misura}
	Per poter raggiungere gli obiettivi prefissati, si è ripetuto l’esperimento nella sua interezza più volte. L’esperimento è stato suddiviso 
in fasi successive per un’ottimizzazione dei risultati e del tempo impiegato.
\begin{description}

  \item[Fase 1] \hfill \\
	  Misurazione delle masse: per prima cosa si sono misurate sulla bilancia di precisione le masse che venivano inserite all’interno 
	  del calorimetro: il recipiente per l’acqua assieme all’agitatore e il cilindro in acciaio. Successivamente è stato riempito il 
	  contenitore con dell’acqua sufficiente a permettere l’immersione totale del cilindro, ed il sistema recipiente-agitatore-acqua è 
	  stato nuovamente pesato per avere una stima della massa dell’acqua utilizzata.
  \item[Fase 2] \hfill \\
	  Preparazione: per preparare l’esperimento per prima cosa si è messo il cilindro a riscaldare nello strumento apposito, dopodiché 
	  si è posto nel calorimetro il recipiente con l’acqua, si è bloccato il coperchio, avvitato l’agitatore al supporto presente sul 
	  tappo del calorimetro, e si e immerso in acqua il termometro (sempre attraverso l’apposito supporto).
  \item[Fase 3] \hfill \\
	Misurazione dei tempi: non appena il riscaldatore ha portato il cilindro a 
	temperatura (valore che dipendeva, di volta in volta, a causa delle condizioni del riscaldatore e delle condizioni ambientali e fluttuazioni casuali), 
	si è agitata l’acqua grazie all’agitatore, e registrata la misura della temperatura mostrata dal termometro immerso in acqua. 
	Successivamente, il più rapidamente possibile, si è preso il cilindro grazie al filo a cui era attaccato e si è immerso nell’acqua
	posta dentro il calorimetro facendolo passare dall’apposito buco, repentinamente chiuso. Un altro sperimentatore ha cercato con 
	maggior precisione possibile di avviare il cronometro nel momento in cui il cilindro si è immerso in acqua. Per limitare gli errori
	di zero il movimento dello sperimentatore che ha spostato il cilindro in acciaio è stato composto di gesti ampi e quindi 
	prevedibile per il secondo sperimentatore, che ha cercato nel miglior modo possibile di riconoscere, nonostante l’assenza del 
	contatto visivo diretto, il momento in cui il cilindro si immergeva in acqua. A intervalli regolari ($5.0 \pm 0.1 $) $s$ è stata 
	registrata la temperatura mostrata dal cronometro immerso in acqua, fino al momento in cui essa non è risultata stabile nel tempo.
	A quel punto si è misurato il tempo segnato dal cronometro durante il cambio della temperatura mostrata dal termometro (secondo la 
	sua scala di riferimento, quindi ogni $0.1 \gradi C$). Preso un numero di misurazioni ritenuto sufficiente si è interrotto l’esperimento 
	aprendo il calorimetro. L’acqua è stata buttata, tutti i componenti sono stati asciugati e si è ripetuto l’esperimento dalla Fase 
	1.
\end{description}



\newpage
\section{Presentazione dei dati}			
	\subsection{Tabelle}
	%\begin{tabella}
%	\centering
%	\input{./tabelle/2030d.tex}
%	\caption{Periodo di oscillazione [s] in posizione dritta, intervalli di 2cm}
%	\label{tab:dec}
%\end{tabella}


	
	\clearpage
	\subsection{Grafici}
	%\begin{grafico}
%    \centering
%\input{../gnuplot/immagini/parabola.tex}
%\caption{Rami di parabola contenenti le due intersezioni}
%\label{img:dec}
%\end{grafico}

\begin{grafico}
    \centering
\begin{tikzpicture}[gnuplot]
%% generated with GNUPLOT 4.6p3 (Lua 5.1; terminal rev. 99, script rev. 100)
%% mar 13 mag 2014 17:23:06 CEST
\gpmonochromelines
\path (0.000,0.000) rectangle (12.500,8.750);
\gpcolor{color=gp lt color border}
\gpsetlinetype{gp lt border}
\gpsetlinewidth{1.00}
\draw[gp path] (1.320,0.985)--(1.500,0.985);
\draw[gp path] (11.947,0.985)--(11.767,0.985);
\node[gp node right] at (1.136,0.985) { 23};
\draw[gp path] (1.320,2.125)--(1.500,2.125);
\draw[gp path] (11.947,2.125)--(11.767,2.125);
\node[gp node right] at (1.136,2.125) { 24};
\draw[gp path] (1.320,3.265)--(1.500,3.265);
\draw[gp path] (11.947,3.265)--(11.767,3.265);
\node[gp node right] at (1.136,3.265) { 25};
\draw[gp path] (1.320,4.405)--(1.500,4.405);
\draw[gp path] (11.947,4.405)--(11.767,4.405);
\node[gp node right] at (1.136,4.405) { 26};
\draw[gp path] (1.320,5.545)--(1.500,5.545);
\draw[gp path] (11.947,5.545)--(11.767,5.545);
\node[gp node right] at (1.136,5.545) { 27};
\draw[gp path] (1.320,6.685)--(1.500,6.685);
\draw[gp path] (11.947,6.685)--(11.767,6.685);
\node[gp node right] at (1.136,6.685) { 28};
\draw[gp path] (1.320,7.825)--(1.500,7.825);
\draw[gp path] (11.947,7.825)--(11.767,7.825);
\node[gp node right] at (1.136,7.825) { 29};
\draw[gp path] (1.320,0.985)--(1.320,1.165);
\draw[gp path] (1.320,7.825)--(1.320,7.645);
\node[gp node center] at (1.320,0.677) { 0};
\draw[gp path] (3.445,0.985)--(3.445,1.165);
\draw[gp path] (3.445,7.825)--(3.445,7.645);
\node[gp node center] at (3.445,0.677) { 50};
\draw[gp path] (5.571,0.985)--(5.571,1.165);
\draw[gp path] (5.571,7.825)--(5.571,7.645);
\node[gp node center] at (5.571,0.677) { 100};
\draw[gp path] (7.696,0.985)--(7.696,1.165);
\draw[gp path] (7.696,7.825)--(7.696,7.645);
\node[gp node center] at (7.696,0.677) { 150};
\draw[gp path] (9.822,0.985)--(9.822,1.165);
\draw[gp path] (9.822,7.825)--(9.822,7.645);
\node[gp node center] at (9.822,0.677) { 200};
\draw[gp path] (11.947,0.985)--(11.947,1.165);
\draw[gp path] (11.947,7.825)--(11.947,7.645);
\node[gp node center] at (11.947,0.677) { 250};
\draw[gp path] (1.320,7.825)--(1.320,0.985)--(11.947,0.985)--(11.947,7.825)--cycle;
\node[gp node center,rotate=-270] at (0.246,4.405) {Temperatura $[\gradi C]$};
\node[gp node center] at (6.633,0.215) {Tempo $[s]$};
\node[gp node center] at (6.633,8.287) {Test 1};
\node[gp node left] at (3.855,1.627) {Dati};
\gpcolor{color=gp lt color 0}
\gpsetpointsize{4.00}
\gppoint{gp mark 1}{(1.533,1.441)}
\gppoint{gp mark 1}{(1.745,4.291)}
\gppoint{gp mark 1}{(1.958,6.229)}
\gppoint{gp mark 1}{(2.170,6.913)}
\gppoint{gp mark 1}{(2.383,7.369)}
\gppoint{gp mark 1}{(2.595,7.483)}
\gppoint{gp mark 1}{(2.808,7.597)}
\gppoint{gp mark 1}{(3.020,7.597)}
\gppoint{gp mark 1}{(3.233,7.597)}
\gppoint{gp mark 1}{(3.445,7.711)}
\gppoint{gp mark 1}{(5.401,7.597)}
\gppoint{gp mark 1}{(7.994,7.483)}
\gppoint{gp mark 1}{(11.437,7.369)}
\gppoint{gp mark 1}{(11.121,1.627)}
\gpcolor{color=gp lt color border}
\node[gp node left] at (3.855,1.319) {Perdita di calore: $f(x) = a[\gradi C] + b [\frac{\gradi C}{s}] \cdot x$};
\gpcolor{color=gp lt color 1}
\gpsetlinetype{gp lt plot 1}
\draw[gp path] (10.663,1.319)--(11.579,1.319);
\draw[gp path] (1.533,7.492)--(1.633,7.491)--(1.733,7.489)--(1.833,7.488)--(1.933,7.487)%
  --(2.033,7.486)--(2.133,7.484)--(2.233,7.483)--(2.333,7.482)--(2.433,7.481)--(2.533,7.479)%
  --(2.633,7.478)--(2.733,7.477)--(2.833,7.476)--(2.933,7.474)--(3.033,7.473)--(3.133,7.472)%
  --(3.233,7.471)--(3.333,7.470)--(3.433,7.468)--(3.533,7.467)--(3.633,7.466)--(3.734,7.465)%
  --(3.834,7.463)--(3.934,7.462)--(4.034,7.461)--(4.134,7.460)--(4.234,7.458)--(4.334,7.457)%
  --(4.434,7.456)--(4.534,7.455)--(4.634,7.454)--(4.734,7.452)--(4.834,7.451)--(4.934,7.450)%
  --(5.034,7.449)--(5.134,7.447)--(5.234,7.446)--(5.334,7.445)--(5.434,7.444)--(5.534,7.442)%
  --(5.634,7.441)--(5.734,7.440)--(5.834,7.439)--(5.934,7.437)--(6.035,7.436)--(6.135,7.435)%
  --(6.235,7.434)--(6.335,7.433)--(6.435,7.431)--(6.535,7.430)--(6.635,7.429)--(6.735,7.428)%
  --(6.835,7.426)--(6.935,7.425)--(7.035,7.424)--(7.135,7.423)--(7.235,7.421)--(7.335,7.420)%
  --(7.435,7.419)--(7.535,7.418)--(7.635,7.416)--(7.735,7.415)--(7.835,7.414)--(7.935,7.413)%
  --(8.035,7.412)--(8.135,7.410)--(8.235,7.409)--(8.336,7.408)--(8.436,7.407)--(8.536,7.405)%
  --(8.636,7.404)--(8.736,7.403)--(8.836,7.402)--(8.936,7.400)--(9.036,7.399)--(9.136,7.398)%
  --(9.236,7.397)--(9.336,7.396)--(9.436,7.394)--(9.536,7.393)--(9.636,7.392)--(9.736,7.391)%
  --(9.836,7.389)--(9.936,7.388)--(10.036,7.387)--(10.136,7.386)--(10.236,7.384)--(10.336,7.383)%
  --(10.436,7.382)--(10.537,7.381)--(10.637,7.379)--(10.737,7.378)--(10.837,7.377)--(10.937,7.376)%
  --(11.037,7.375)--(11.137,7.373)--(11.237,7.372)--(11.337,7.371)--(11.437,7.370);
\gpcolor{color=gp lt color border}
\gpsetlinetype{gp lt border}
\draw[gp path] (1.320,7.825)--(1.320,0.985)--(11.947,0.985)--(11.947,7.825)--cycle;
%% coordinates of the plot area
\gpdefrectangularnode{gp plot 1}{\pgfpoint{1.320cm}{0.985cm}}{\pgfpoint{11.947cm}{7.825cm}}
\end{tikzpicture}
%% gnuplot variables

\caption{Primo Test: $\quad y = (28.71 \pm 0.01) + (-0.00046 \pm 0.00003) x$}
\label{img:t1}
\end{grafico}

%\[ y = (28.71 \pm 0.01) + (-0.00046 \pm 0.00003) x\]

\begin{grafico}
    \centering
\begin{tikzpicture}[gnuplot]
%% generated with GNUPLOT 4.6p3 (Lua 5.1; terminal rev. 99, script rev. 100)
%% mar 13 mag 2014 17:23:06 CEST
\gpmonochromelines
\path (0.000,0.000) rectangle (12.500,8.750);
\gpcolor{color=gp lt color border}
\gpsetlinetype{gp lt border}
\gpsetlinewidth{1.00}
\draw[gp path] (1.320,0.985)--(1.500,0.985);
\draw[gp path] (11.947,0.985)--(11.767,0.985);
\node[gp node right] at (1.136,0.985) { 22};
\draw[gp path] (1.320,2.125)--(1.500,2.125);
\draw[gp path] (11.947,2.125)--(11.767,2.125);
\node[gp node right] at (1.136,2.125) { 23};
\draw[gp path] (1.320,3.265)--(1.500,3.265);
\draw[gp path] (11.947,3.265)--(11.767,3.265);
\node[gp node right] at (1.136,3.265) { 24};
\draw[gp path] (1.320,4.405)--(1.500,4.405);
\draw[gp path] (11.947,4.405)--(11.767,4.405);
\node[gp node right] at (1.136,4.405) { 25};
\draw[gp path] (1.320,5.545)--(1.500,5.545);
\draw[gp path] (11.947,5.545)--(11.767,5.545);
\node[gp node right] at (1.136,5.545) { 26};
\draw[gp path] (1.320,6.685)--(1.500,6.685);
\draw[gp path] (11.947,6.685)--(11.767,6.685);
\node[gp node right] at (1.136,6.685) { 27};
\draw[gp path] (1.320,7.825)--(1.500,7.825);
\draw[gp path] (11.947,7.825)--(11.767,7.825);
\node[gp node right] at (1.136,7.825) { 28};
\draw[gp path] (1.320,0.985)--(1.320,1.165);
\draw[gp path] (1.320,7.825)--(1.320,7.645);
\node[gp node center] at (1.320,0.677) { 0};
\draw[gp path] (2.838,0.985)--(2.838,1.165);
\draw[gp path] (2.838,7.825)--(2.838,7.645);
\node[gp node center] at (2.838,0.677) { 50};
\draw[gp path] (4.356,0.985)--(4.356,1.165);
\draw[gp path] (4.356,7.825)--(4.356,7.645);
\node[gp node center] at (4.356,0.677) { 100};
\draw[gp path] (5.874,0.985)--(5.874,1.165);
\draw[gp path] (5.874,7.825)--(5.874,7.645);
\node[gp node center] at (5.874,0.677) { 150};
\draw[gp path] (7.393,0.985)--(7.393,1.165);
\draw[gp path] (7.393,7.825)--(7.393,7.645);
\node[gp node center] at (7.393,0.677) { 200};
\draw[gp path] (8.911,0.985)--(8.911,1.165);
\draw[gp path] (8.911,7.825)--(8.911,7.645);
\node[gp node center] at (8.911,0.677) { 250};
\draw[gp path] (10.429,0.985)--(10.429,1.165);
\draw[gp path] (10.429,7.825)--(10.429,7.645);
\node[gp node center] at (10.429,0.677) { 300};
\draw[gp path] (11.947,0.985)--(11.947,1.165);
\draw[gp path] (11.947,7.825)--(11.947,7.645);
\node[gp node center] at (11.947,0.677) { 350};
\draw[gp path] (1.320,7.825)--(1.320,0.985)--(11.947,0.985)--(11.947,7.825)--cycle;
\node[gp node center,rotate=-270] at (0.246,4.405) {Temperatura $[\gradi C]$};
\node[gp node center] at (6.633,0.215) {Tempo $[s]$};
\node[gp node center] at (6.633,8.287) {Test 2};
\node[gp node left] at (3.855,1.627) {Dati};
\gpcolor{color=gp lt color 0}
\gpsetpointsize{4.00}
\gppoint{gp mark 1}{(1.472,1.555)}
\gppoint{gp mark 1}{(1.624,3.835)}
\gppoint{gp mark 1}{(1.775,5.431)}
\gppoint{gp mark 1}{(1.927,6.229)}
\gppoint{gp mark 1}{(2.079,6.571)}
\gppoint{gp mark 1}{(2.231,7.255)}
\gppoint{gp mark 1}{(2.383,7.369)}
\gppoint{gp mark 1}{(2.535,7.483)}
\gppoint{gp mark 1}{(2.686,7.597)}
\gppoint{gp mark 1}{(2.838,7.711)}
\gppoint{gp mark 1}{(6.603,7.597)}
\gppoint{gp mark 1}{(9.093,7.483)}
\gppoint{gp mark 1}{(11.704,7.369)}
\gppoint{gp mark 1}{(11.121,1.627)}
\gpcolor{color=gp lt color border}
\node[gp node left] at (3.855,1.319) {Perdita di calore: $f(x) = a[\gradi C] + b [\frac{\gradi C}{s}] \cdot x$};
\gpcolor{color=gp lt color 1}
\gpsetlinetype{gp lt plot 1}
\draw[gp path] (10.663,1.319)--(11.579,1.319);
\draw[gp path] (1.472,7.774)--(1.575,7.770)--(1.679,7.766)--(1.782,7.762)--(1.885,7.758)%
  --(1.989,7.754)--(2.092,7.750)--(2.195,7.746)--(2.299,7.742)--(2.402,7.738)--(2.505,7.734)%
  --(2.609,7.730)--(2.712,7.726)--(2.815,7.722)--(2.919,7.718)--(3.022,7.714)--(3.126,7.710)%
  --(3.229,7.706)--(3.332,7.702)--(3.436,7.698)--(3.539,7.694)--(3.642,7.690)--(3.746,7.686)%
  --(3.849,7.682)--(3.952,7.678)--(4.056,7.674)--(4.159,7.670)--(4.262,7.666)--(4.366,7.662)%
  --(4.469,7.658)--(4.573,7.654)--(4.676,7.650)--(4.779,7.646)--(4.883,7.642)--(4.986,7.638)%
  --(5.089,7.634)--(5.193,7.630)--(5.296,7.626)--(5.399,7.622)--(5.503,7.618)--(5.606,7.614)%
  --(5.709,7.610)--(5.813,7.606)--(5.916,7.602)--(6.019,7.598)--(6.123,7.594)--(6.226,7.590)%
  --(6.330,7.586)--(6.433,7.582)--(6.536,7.578)--(6.640,7.574)--(6.743,7.570)--(6.846,7.566)%
  --(6.950,7.562)--(7.053,7.558)--(7.156,7.554)--(7.260,7.550)--(7.363,7.546)--(7.466,7.542)%
  --(7.570,7.538)--(7.673,7.534)--(7.777,7.530)--(7.880,7.526)--(7.983,7.522)--(8.087,7.518)%
  --(8.190,7.514)--(8.293,7.510)--(8.397,7.506)--(8.500,7.502)--(8.603,7.498)--(8.707,7.494)%
  --(8.810,7.490)--(8.913,7.486)--(9.017,7.482)--(9.120,7.478)--(9.224,7.474)--(9.327,7.470)%
  --(9.430,7.466)--(9.534,7.462)--(9.637,7.458)--(9.740,7.454)--(9.844,7.450)--(9.947,7.446)%
  --(10.050,7.442)--(10.154,7.438)--(10.257,7.434)--(10.360,7.430)--(10.464,7.426)--(10.567,7.422)%
  --(10.671,7.418)--(10.774,7.414)--(10.877,7.410)--(10.981,7.406)--(11.084,7.402)--(11.187,7.398)%
  --(11.291,7.394)--(11.394,7.390)--(11.497,7.386)--(11.601,7.382)--(11.704,7.378);
\gpcolor{color=gp lt color border}
\gpsetlinetype{gp lt border}
\draw[gp path] (1.320,7.825)--(1.320,0.985)--(11.947,0.985)--(11.947,7.825)--cycle;
%% coordinates of the plot area
\gpdefrectangularnode{gp plot 1}{\pgfpoint{1.320cm}{0.985cm}}{\pgfpoint{11.947cm}{7.825cm}}
\end{tikzpicture}
%% gnuplot variables

\caption[]{Secondo Test: $\quad y = (27.96 \pm 0.01) + (-0.00103 \pm 0.00007)  x$}
\label{img:t2}
\end{grafico}

\begin{grafico}
    \centering
\begin{tikzpicture}[gnuplot]
%% generated with GNUPLOT 4.6p3 (Lua 5.1; terminal rev. 99, script rev. 100)
%% mar 13 mag 2014 18:41:53 CEST
\gpmonochromelines
\path (0.000,0.000) rectangle (12.500,8.750);
\gpcolor{color=gp lt color border}
\gpsetlinetype{gp lt border}
\gpsetlinewidth{1.00}
\draw[gp path] (1.320,0.985)--(1.500,0.985);
\draw[gp path] (11.947,0.985)--(11.767,0.985);
\node[gp node right] at (1.136,0.985) { 23};
\draw[gp path] (1.320,1.840)--(1.500,1.840);
\draw[gp path] (11.947,1.840)--(11.767,1.840);
\node[gp node right] at (1.136,1.840) { 24};
\draw[gp path] (1.320,2.695)--(1.500,2.695);
\draw[gp path] (11.947,2.695)--(11.767,2.695);
\node[gp node right] at (1.136,2.695) { 25};
\draw[gp path] (1.320,3.550)--(1.500,3.550);
\draw[gp path] (11.947,3.550)--(11.767,3.550);
\node[gp node right] at (1.136,3.550) { 26};
\draw[gp path] (1.320,4.405)--(1.500,4.405);
\draw[gp path] (11.947,4.405)--(11.767,4.405);
\node[gp node right] at (1.136,4.405) { 27};
\draw[gp path] (1.320,5.260)--(1.500,5.260);
\draw[gp path] (11.947,5.260)--(11.767,5.260);
\node[gp node right] at (1.136,5.260) { 28};
\draw[gp path] (1.320,6.115)--(1.500,6.115);
\draw[gp path] (11.947,6.115)--(11.767,6.115);
\node[gp node right] at (1.136,6.115) { 29};
\draw[gp path] (1.320,6.970)--(1.500,6.970);
\draw[gp path] (11.947,6.970)--(11.767,6.970);
\node[gp node right] at (1.136,6.970) { 30};
\draw[gp path] (1.320,7.825)--(1.500,7.825);
\draw[gp path] (11.947,7.825)--(11.767,7.825);
\node[gp node right] at (1.136,7.825) { 31};
\draw[gp path] (1.320,0.985)--(1.320,1.165);
\draw[gp path] (1.320,7.825)--(1.320,7.645);
\node[gp node center] at (1.320,0.677) { 0};
\draw[gp path] (2.648,0.985)--(2.648,1.165);
\draw[gp path] (2.648,7.825)--(2.648,7.645);
\node[gp node center] at (2.648,0.677) { 50};
\draw[gp path] (3.977,0.985)--(3.977,1.165);
\draw[gp path] (3.977,7.825)--(3.977,7.645);
\node[gp node center] at (3.977,0.677) { 100};
\draw[gp path] (5.305,0.985)--(5.305,1.165);
\draw[gp path] (5.305,7.825)--(5.305,7.645);
\node[gp node center] at (5.305,0.677) { 150};
\draw[gp path] (6.634,0.985)--(6.634,1.165);
\draw[gp path] (6.634,7.825)--(6.634,7.645);
\node[gp node center] at (6.634,0.677) { 200};
\draw[gp path] (7.962,0.985)--(7.962,1.165);
\draw[gp path] (7.962,7.825)--(7.962,7.645);
\node[gp node center] at (7.962,0.677) { 250};
\draw[gp path] (9.290,0.985)--(9.290,1.165);
\draw[gp path] (9.290,7.825)--(9.290,7.645);
\node[gp node center] at (9.290,0.677) { 300};
\draw[gp path] (10.619,0.985)--(10.619,1.165);
\draw[gp path] (10.619,7.825)--(10.619,7.645);
\node[gp node center] at (10.619,0.677) { 350};
\draw[gp path] (11.947,0.985)--(11.947,1.165);
\draw[gp path] (11.947,7.825)--(11.947,7.645);
\node[gp node center] at (11.947,0.677) { 400};
\draw[gp path] (1.320,7.825)--(1.320,0.985)--(11.947,0.985)--(11.947,7.825)--cycle;
\node[gp node center,rotate=-270] at (0.246,4.405) {Temperatura $[\gradi C]$};
\node[gp node center] at (6.633,0.215) {Tempo $[s]$};
\node[gp node center] at (6.633,8.287) {Test 3};
\node[gp node left] at (3.855,1.627) {Dati};
\gpcolor{color=gp lt color 0}
\gpsetpointsize{4.00}
\gppoint{gp mark 1}{(1.453,1.413)}
\gppoint{gp mark 1}{(1.586,3.037)}
\gppoint{gp mark 1}{(1.719,3.978)}
\gppoint{gp mark 1}{(1.851,4.918)}
\gppoint{gp mark 1}{(1.984,5.431)}
\gppoint{gp mark 1}{(2.117,5.773)}
\gppoint{gp mark 1}{(2.250,6.029)}
\gppoint{gp mark 1}{(2.383,6.201)}
\gppoint{gp mark 1}{(2.516,6.372)}
\gppoint{gp mark 1}{(2.648,6.543)}
\gppoint{gp mark 1}{(2.781,6.628)}
\gppoint{gp mark 1}{(2.914,6.713)}
\gppoint{gp mark 1}{(3.047,6.713)}
\gppoint{gp mark 1}{(3.180,6.799)}
\gppoint{gp mark 1}{(3.313,6.799)}
\gppoint{gp mark 1}{(3.445,6.799)}
\gppoint{gp mark 1}{(3.578,6.884)}
\gppoint{gp mark 1}{(6.288,6.799)}
\gppoint{gp mark 1}{(7.776,6.713)}
\gppoint{gp mark 1}{(9.343,6.628)}
\gppoint{gp mark 1}{(10.911,6.543)}
\gppoint{gp mark 1}{(11.121,1.627)}
\gpcolor{color=gp lt color border}
\node[gp node left] at (3.855,1.319) {Perdita di calore: $f(x) = a[\gradi C] + b [\frac{\gradi C}{s}] \cdot x$};
\gpcolor{color=gp lt color 1}
\gpsetlinetype{gp lt plot 1}
\draw[gp path] (10.663,1.319)--(11.579,1.319);
\draw[gp path] (1.453,6.998)--(1.548,6.993)--(1.644,6.989)--(1.739,6.984)--(1.835,6.979)%
  --(1.931,6.975)--(2.026,6.970)--(2.122,6.966)--(2.217,6.961)--(2.313,6.956)--(2.408,6.952)%
  --(2.504,6.947)--(2.599,6.942)--(2.695,6.938)--(2.790,6.933)--(2.886,6.929)--(2.981,6.924)%
  --(3.077,6.919)--(3.172,6.915)--(3.268,6.910)--(3.364,6.906)--(3.459,6.901)--(3.555,6.896)%
  --(3.650,6.892)--(3.746,6.887)--(3.841,6.882)--(3.937,6.878)--(4.032,6.873)--(4.128,6.869)%
  --(4.223,6.864)--(4.319,6.859)--(4.414,6.855)--(4.510,6.850)--(4.606,6.846)--(4.701,6.841)%
  --(4.797,6.836)--(4.892,6.832)--(4.988,6.827)--(5.083,6.823)--(5.179,6.818)--(5.274,6.813)%
  --(5.370,6.809)--(5.465,6.804)--(5.561,6.799)--(5.656,6.795)--(5.752,6.790)--(5.847,6.786)%
  --(5.943,6.781)--(6.039,6.776)--(6.134,6.772)--(6.230,6.767)--(6.325,6.763)--(6.421,6.758)%
  --(6.516,6.753)--(6.612,6.749)--(6.707,6.744)--(6.803,6.740)--(6.898,6.735)--(6.994,6.730)%
  --(7.089,6.726)--(7.185,6.721)--(7.281,6.716)--(7.376,6.712)--(7.472,6.707)--(7.567,6.703)%
  --(7.663,6.698)--(7.758,6.693)--(7.854,6.689)--(7.949,6.684)--(8.045,6.680)--(8.140,6.675)%
  --(8.236,6.670)--(8.331,6.666)--(8.427,6.661)--(8.522,6.657)--(8.618,6.652)--(8.714,6.647)%
  --(8.809,6.643)--(8.905,6.638)--(9.000,6.633)--(9.096,6.629)--(9.191,6.624)--(9.287,6.620)%
  --(9.382,6.615)--(9.478,6.610)--(9.573,6.606)--(9.669,6.601)--(9.764,6.597)--(9.860,6.592)%
  --(9.956,6.587)--(10.051,6.583)--(10.147,6.578)--(10.242,6.574)--(10.338,6.569)--(10.433,6.564)%
  --(10.529,6.560)--(10.624,6.555)--(10.720,6.550)--(10.815,6.546)--(10.911,6.541);
\gpcolor{color=gp lt color border}
\gpsetlinetype{gp lt border}
\draw[gp path] (1.320,7.825)--(1.320,0.985)--(11.947,0.985)--(11.947,7.825)--cycle;
%% coordinates of the plot area
\gpdefrectangularnode{gp plot 1}{\pgfpoint{1.320cm}{0.985cm}}{\pgfpoint{11.947cm}{7.825cm}}
\end{tikzpicture}
%% gnuplot variables

\caption[]{Terzo Test: $\quad  y = (30.04 \pm 0.03) + (-0.0015 \pm 0.0001)  x$}
\label{img:t3}
\end{grafico}

\begin{grafico}
    \centering
\begin{tikzpicture}[gnuplot]
%% generated with GNUPLOT 4.6p3 (Lua 5.1; terminal rev. 99, script rev. 100)
%% mar 13 mag 2014 15:56:53 CEST
\gpmonochromelines
\path (0.000,0.000) rectangle (12.500,8.750);
\gpcolor{color=gp lt color border}
\gpsetlinetype{gp lt border}
\gpsetlinewidth{1.00}
\draw[gp path] (1.320,0.985)--(1.500,0.985);
\draw[gp path] (11.947,0.985)--(11.767,0.985);
\node[gp node right] at (1.136,0.985) { 23};
\draw[gp path] (1.320,2.125)--(1.500,2.125);
\draw[gp path] (11.947,2.125)--(11.767,2.125);
\node[gp node right] at (1.136,2.125) { 24};
\draw[gp path] (1.320,3.265)--(1.500,3.265);
\draw[gp path] (11.947,3.265)--(11.767,3.265);
\node[gp node right] at (1.136,3.265) { 25};
\draw[gp path] (1.320,4.405)--(1.500,4.405);
\draw[gp path] (11.947,4.405)--(11.767,4.405);
\node[gp node right] at (1.136,4.405) { 26};
\draw[gp path] (1.320,5.545)--(1.500,5.545);
\draw[gp path] (11.947,5.545)--(11.767,5.545);
\node[gp node right] at (1.136,5.545) { 27};
\draw[gp path] (1.320,6.685)--(1.500,6.685);
\draw[gp path] (11.947,6.685)--(11.767,6.685);
\node[gp node right] at (1.136,6.685) { 28};
\draw[gp path] (1.320,7.825)--(1.500,7.825);
\draw[gp path] (11.947,7.825)--(11.767,7.825);
\node[gp node right] at (1.136,7.825) { 29};
\draw[gp path] (1.320,0.985)--(1.320,1.165);
\draw[gp path] (1.320,7.825)--(1.320,7.645);
\node[gp node center] at (1.320,0.677) { 0};
\draw[gp path] (3.445,0.985)--(3.445,1.165);
\draw[gp path] (3.445,7.825)--(3.445,7.645);
\node[gp node center] at (3.445,0.677) { 50};
\draw[gp path] (5.571,0.985)--(5.571,1.165);
\draw[gp path] (5.571,7.825)--(5.571,7.645);
\node[gp node center] at (5.571,0.677) { 100};
\draw[gp path] (7.696,0.985)--(7.696,1.165);
\draw[gp path] (7.696,7.825)--(7.696,7.645);
\node[gp node center] at (7.696,0.677) { 150};
\draw[gp path] (9.822,0.985)--(9.822,1.165);
\draw[gp path] (9.822,7.825)--(9.822,7.645);
\node[gp node center] at (9.822,0.677) { 200};
\draw[gp path] (11.947,0.985)--(11.947,1.165);
\draw[gp path] (11.947,7.825)--(11.947,7.645);
\node[gp node center] at (11.947,0.677) { 250};
\draw[gp path] (1.320,7.825)--(1.320,0.985)--(11.947,0.985)--(11.947,7.825)--cycle;
\node[gp node center,rotate=-270] at (0.246,4.405) {Temperatura $[\gradi C]$};
\node[gp node center] at (6.633,0.215) {Tempo $[s]$};
\node[gp node center] at (6.633,8.287) {Test 4};
\node[gp node left] at (3.855,1.627) {Dati};
\gpcolor{color=gp lt color 0}
\gpsetpointsize{4.00}
\gppoint{gp mark 1}{(1.533,1.441)}
\gppoint{gp mark 1}{(1.745,4.291)}
\gppoint{gp mark 1}{(1.958,6.229)}
\gppoint{gp mark 1}{(2.170,6.913)}
\gppoint{gp mark 1}{(2.383,7.369)}
\gppoint{gp mark 1}{(2.595,7.483)}
\gppoint{gp mark 1}{(2.808,7.597)}
\gppoint{gp mark 1}{(3.020,7.597)}
\gppoint{gp mark 1}{(3.233,7.597)}
\gppoint{gp mark 1}{(3.445,7.711)}
\gppoint{gp mark 1}{(5.401,7.597)}
\gppoint{gp mark 1}{(7.994,7.483)}
\gppoint{gp mark 1}{(11.437,7.369)}
\gppoint{gp mark 1}{(11.121,1.627)}
\gpcolor{color=gp lt color border}
\node[gp node left] at (3.855,1.319) {Perdita di calore: $f(x) = a[\gradi C] + b [\frac{\gradi C}{s}] \cdot x$};
\gpcolor{color=gp lt color 1}
\gpsetlinetype{gp lt plot 1}
\draw[gp path] (10.663,1.319)--(11.579,1.319);
\draw[gp path] (1.533,7.771)--(1.633,7.767)--(1.733,7.763)--(1.833,7.759)--(1.933,7.755)%
  --(2.033,7.751)--(2.133,7.747)--(2.233,7.743)--(2.333,7.739)--(2.433,7.735)--(2.533,7.731)%
  --(2.633,7.727)--(2.733,7.723)--(2.833,7.719)--(2.933,7.715)--(3.033,7.710)--(3.133,7.706)%
  --(3.233,7.702)--(3.333,7.698)--(3.433,7.694)--(3.533,7.690)--(3.633,7.686)--(3.734,7.682)%
  --(3.834,7.678)--(3.934,7.674)--(4.034,7.670)--(4.134,7.666)--(4.234,7.662)--(4.334,7.658)%
  --(4.434,7.654)--(4.534,7.650)--(4.634,7.646)--(4.734,7.642)--(4.834,7.638)--(4.934,7.634)%
  --(5.034,7.630)--(5.134,7.626)--(5.234,7.622)--(5.334,7.618)--(5.434,7.614)--(5.534,7.610)%
  --(5.634,7.606)--(5.734,7.602)--(5.834,7.598)--(5.934,7.594)--(6.035,7.590)--(6.135,7.586)%
  --(6.235,7.582)--(6.335,7.578)--(6.435,7.574)--(6.535,7.570)--(6.635,7.566)--(6.735,7.562)%
  --(6.835,7.558)--(6.935,7.554)--(7.035,7.549)--(7.135,7.545)--(7.235,7.541)--(7.335,7.537)%
  --(7.435,7.533)--(7.535,7.529)--(7.635,7.525)--(7.735,7.521)--(7.835,7.517)--(7.935,7.513)%
  --(8.035,7.509)--(8.135,7.505)--(8.235,7.501)--(8.336,7.497)--(8.436,7.493)--(8.536,7.489)%
  --(8.636,7.485)--(8.736,7.481)--(8.836,7.477)--(8.936,7.473)--(9.036,7.469)--(9.136,7.465)%
  --(9.236,7.461)--(9.336,7.457)--(9.436,7.453)--(9.536,7.449)--(9.636,7.445)--(9.736,7.441)%
  --(9.836,7.437)--(9.936,7.433)--(10.036,7.429)--(10.136,7.425)--(10.236,7.421)--(10.336,7.417)%
  --(10.436,7.413)--(10.537,7.409)--(10.637,7.405)--(10.737,7.401)--(10.837,7.397)--(10.937,7.393)%
  --(11.037,7.389)--(11.137,7.384)--(11.237,7.380)--(11.337,7.376)--(11.437,7.372);
\gpcolor{color=gp lt color border}
\gpsetlinetype{gp lt border}
\draw[gp path] (1.320,7.825)--(1.320,0.985)--(11.947,0.985)--(11.947,7.825)--cycle;
%% coordinates of the plot area
\gpdefrectangularnode{gp plot 1}{\pgfpoint{1.320cm}{0.985cm}}{\pgfpoint{11.947cm}{7.825cm}}
\end{tikzpicture}
%% gnuplot variables

\caption[]{Quarto Test: $\quad  y = (28.96 \pm 0.02) + (-0.0015 \pm 0.0001)  x$}
\label{img:t4}
\end{grafico}


		
\section{Conclusioni}
	L'esperimento ha dato vita a un valore abbastanza accettabile, che va a confermare le previsioni teoriche (circa $10^{-1}
 cal/g$). L'errore è risultato abbastanza alto, probabilmente a causa di imprecisioni effettuate in ambito sperimentale. Per migliorare
 l'esperimento sarebbe stato necessario ridurre ulteriormente gli errori legati alla dissipazione del calore. 

	
\section{Codice}
	Qui ci sono i programmi usati per l'analisi dei dati

\begin{verbatim}
Calcolo del Cx


#include <iostream>
#include <cmath>
int main ()
{
	using namespace std;
	double Cx;
	double sCx;
	double mx;
	double smx;
	double ma;
	double sma;
	double Ca;
	double mr;
	double smr;
	double Cr;
	double sCr;
	double Kter;
	double sKter;
	double ta;
	double sta;
	double tf;
	double stf;
	double t0;
	double st0;
	cout << "Inserire massa agitatore e vaso con relativo errore: ";
	cin >> mr >> smr;
	cout << "Inserire massa alluminio con relativo errore: ";
	cin >> mx >> smx;
	cout << "Inserire calore specifico acqua ";
	cin >> Ca;
	cout << "Inserire massa acqua con relativo errore ";
	cin >> ma >> sma;
	cout << "Inserire calore specifico rame stagnato con relativo errore: ";
	cin >> Cr >> sCr;
	cout << "Inserire capacità termica del termometro con relativo errore: 
";
	cin >> Kter >> sKter;
	cout << "Inserire la temperatura dell'aceua con relativo errore: ";
	cin >> ta >> sta;
	cout << "Inserire temperatura iniziale con relativo errore: ";
	cin >> t0 >> st0;
	cout << "Inserire temperatura finale con relativo errore: ";
	cin >> tf >> stf;
	Cx = ( ( (ma * Ca ) + (mr * Cr) + Kter ) * abs(tf - ta) ) / ( mx * 
abs(tf -t0) );
	double ddma = ( Ca * abs(tf - ta) ) / ( mx * abs(tf - t0));
	double ddmr = ( Cr * abs(tf - ta) ) / ( mx * abs(tf - t0));
	double ddCr = ( mr * abs(tf - ta) ) / ( mx * abs(tf - t0));
	double ddKter = abs(tf - ta) / ( mx * abs(tf - t0));
	double ddtf =  ( ( (ma * Ca) + (mr * Cr) + Kter ) * (ta - t0) ) / (mx * 
(t0 - tf) * (t0 - tf) );
	double ddta = - ( (ma *Ca) + (mr * Cr) + Kter ) / ( mx * ( tf - t0));
	double ddmx = - ( ( (ma * Ca ) + (mr * Cr) + Kter ) * abs(tf - ta) ) / ( 
mx *  mx * abs(tf -t0) );
	double ddt0 = ( ( (ma * Ca ) + (mr * Cr) + Kter ) * abs(tf - ta) * mx ) 
/ ( mx * mx * (tf - t0) * (tf - t0));
	sCx = sqrt ((ddma * ddma * sma * sma) +
			(ddmr * ddmr * smr * smr) + (ddKter * ddKter * sKter * 
sKter) +
			(ddCr * ddCr * sCr * sCr) + (ddtf * ddtf * stf * stf) +
			(ddta * ddta * sta * sta) + (ddmx * ddmx * smx * smx) + 
(ddt0 * ddt0 * st0 * st0) );
	cout << Cx << " pm " << sCx;
	return 0;
}



Media Pesata
 
//
//  Created by Simone Frau on 22/03/14.
//
//

#include <iostream>
#include <cmath>
#include <fstream>
#include <cstdlib>
#include <string>

using namespace std;

int main ()
{
    int n;
    cout << "dire di quanti valor si vuole calcolare la media" << endl;
    cin >> n;
    double* misure = new double [n];
    for (int i = 0 ; i < n ; i++)
    {
        cout << "inserire valore "<< i << " della media" << endl;
        cin >> misure[i];
    }

    double* sigme = new double [n];
    for (int i = 0 ; i < n ; i++)
    {
        cout << "inserire valore "<< i << " della sigma" << endl;
        cin >> sigme[i];
    }
    
    double sommavalsig;
    for (int i = 0 ; i < n ; i++)
    {
        sommavalsig += (misure[i]/sigme[i]);
    }
    
    double sommak;
    for (int i = 0 ; i < n ; i++)
    {
        sommak += (1/sigme[i]);
    }
    
    double mediapesata;
    mediapesata = (1/sommak)*(sommavalsig);
    
    cout << " Media pesata: " << mediapesata << endl;
    
    double errormediapesata;
    errormediapesata = sqrt(1/sommak);
    
    cout << " Error media pesata: " << errormediapesata << endl;
    
    return 0;
}

Media, Scarto, Sigma

#include <dirent.h>
#include <iostream>
#include <cmath>
#include <fstream>
#include <cstdlib>
#include <string>

using namespace std;

int main ()
{

    string path = "/Users/FrodoFrau/Documents/Università/Fisica Sperimentale 
teoria/Laboratorio/DATI/";
    
    DIR *dir = opendir( path.c_str() );
    if (!dir)
    {
        cerr << "Dir not fnd" << endl;
        exit(1);
    }
    dirent *entry;
    
    while ((entry = readdir(dir)))
    {
        cout << "Found directory entry: "
        << entry->d_name << endl;
    }
    cout << "Di quale file vuoi fare la Media Aritmetica?" << endl;
    
    string filename;
    getline (cin, filename);
    bool flag = 0;
    closedir(dir);
    
    
    path += filename;
    
    int dimensione = 0;
    double valori;
    
    fstream fin ( path.c_str(), fstream::in );

    while (fin >> valori )
    {
        dimensione++;
    }
    fin.close();
    
    fin.open( path.c_str(), fstream::in );
    
    double* misure = new double [dimensione];
    for( int i = 0 ; i < dimensione ; i++ )
    {
        fin>>misure[i];
    }
    
    double sommatoria;
    for (int i = 0 ; i < dimensione ; i++)
    {
        sommatoria +=  misure[i];
    }
    double media = sommatoria/dimensione;
    cout << "Media: " << media << endl;
    
    double differenza;
    for (int i = 0 ; i < dimensione ; i++)
    {
        differenza += (misure[i]-media) * (misure[i]-media);
    }
    double Scarto = sqrt(differenza/(dimensione));
    cout << "Scarto: " << Scarto << endl;
    
    double Sigma = sqrt(differenza/(dimensione-1));
    cout << "Sigma " << Sigma << endl;
    
    double ErroreScarto = Scarto/sqrt(dimensione);
    cout << "ErroreMedia " << ErroreScarto << endl;
    
    return 0;
}


Compatibilità.h

//
//  Created by Simone Frau & Chiappara Davide on 13/01/14.
//
//

#ifndef _Compatibilita__h
#define _Compatibilita__h

#include <dirent.h>
#include <iostream>
#include <cmath>
#include <fstream>
#include <cstdlib>
#include <string>

using namespace std;


void Compatibilità ()
{
    
    double media_a;
    double media_b;
    double sigma_a;
    double sigma_b;
    
    cout << "inserire prima media" << endl;
    cin >> media_a;
    cout << "inserire seconda media" << endl;
    cin >> media_b;
    cout << "inserire primo sigma" << endl;
    cin >> sigma_a;
    cout << "inserire secondo sigma" << endl;
    cin >> sigma_b;
    double Compatibilità = ( abs ( media_a - media_b ) / ( sqrt ( sigma_a * 
sigma_a + sigma_b * sigma_b ) ) );
    
    
    cout << Compatibilità << endl;
    
    
}
            
        
#endif


Interpolazione

#include <iostream>
#include <cmath>

using namespace std;

int main ()
{
	cout << "Si digitino il numero delle coppie di cui fare la 
interpolazione lineare" << endl;
	int dimensione;
	cin >> dimensione;
	double *misurex = new double [dimensione];
	cout << "Si inseriscano le x" << endl;
	for (int i=0 ; i < dimensione ; i++ )
	{
		cin >> misurex[i];
	}
    double *misurey = new double [dimensione];
	cout << "Si inseriscano le y" << endl;
		for (int i=0 ; i < dimensione ; i++ )
	{
		cin >> misurey[i];
	}
	double delta;
	double sommaQUADx = 0;
	for (int i = 0 ; i < dimensione ; i++)
	{
		sommaQUADx += ( misurex[i] * misurex[i] );
	}
	double sommax = 0;
	for ( int i = 0 ; i < dimensione ; i ++)
	{
		sommax += misurex[i];
	}
	double sommaxQUAD = sommax * sommax;
	delta = (dimensione * sommaQUADx) - sommaxQUAD;
	double a,b;
	double sommay = 0;
	for ( int i = 0 ; i < dimensione ; i ++)
	{
		sommay += misurey[i];
	}
	double sommaxy = 0;
	for ( int i = 0 ; i < dimensione ; i ++)
	{
		sommaxy += misurey[i] * misurex[i];
	}
	a = ( 1 / delta ) * ( ( sommaQUADx * sommay ) - ( sommax * sommaxy ) );
	b = ( 1 / delta ) * ( ( ( dimensione * ( sommaxy ) ) - ( sommax * sommay 
) ));
	double sigmay;
	double sommasigma;
	for ( int i = 0 ; i < dimensione ; i++ )
	{
		sommasigma += ( (a + ( b * misurex[i] ) -misurey[i] ) * (a + ( b 
* misurex[i] ) - misurey[i] ) );
	}
	sigmay = sqrt ( sommasigma / ( dimensione -2 ) );
    
    double sigmab = (sqrt ( dimensione/delta )) * sigmay;
    double sigmaa = (sqrt ( sommaQUADx/delta )) * sigmay;

    
	cout << "La a vale: " << a << endl;
	cout << "La b vale: " << b << endl;
	cout << "La sigma vale: " << sigmay << endl;
    cout << "La sigma di b vale " << sigmab << endl;
    cout << "La sigma di a vale " << sigmaa << endl;

                       
	return 0;
}

\end{verbatim}

	
%\subsection{Esempio immagini}
%\begin{figure}[p]
% \centering
% \includegraphics[width=0.8\textwidth]{spazio1}
% \caption{Spazio!}
% \label{fig:spazio1}
%\end{figure}


\end{document}
