Qui ci sono i programmi usati per l'analisi dei dati

\begin{verbatim}
Calcolo del Cx


#include <iostream>
#include <cmath>
int main ()
{
	using namespace std;
	double Cx;
	double sCx;
	double mx;
	double smx;
	double ma;
	double sma;
	double Ca;
	double mr;
	double smr;
	double Cr;
	double sCr;
	double Kter;
	double sKter;
	double ta;
	double sta;
	double tf;
	double stf;
	double t0;
	double st0;
	cout << "Inserire massa agitatore e vaso con relativo errore: ";
	cin >> mr >> smr;
	cout << "Inserire massa alluminio con relativo errore: ";
	cin >> mx >> smx;
	cout << "Inserire calore specifico acqua ";
	cin >> Ca;
	cout << "Inserire massa acqua con relativo errore ";
	cin >> ma >> sma;
	cout << "Inserire calore specifico rame stagnato con relativo errore: ";
	cin >> Cr >> sCr;
	cout << "Inserire capacità termica del termometro con relativo errore: 
";
	cin >> Kter >> sKter;
	cout << "Inserire la temperatura dell'aceua con relativo errore: ";
	cin >> ta >> sta;
	cout << "Inserire temperatura iniziale con relativo errore: ";
	cin >> t0 >> st0;
	cout << "Inserire temperatura finale con relativo errore: ";
	cin >> tf >> stf;
	Cx = ( ( (ma * Ca ) + (mr * Cr) + Kter ) * abs(tf - ta) ) / ( mx * 
abs(tf -t0) );
	double ddma = ( Ca * abs(tf - ta) ) / ( mx * abs(tf - t0));
	double ddmr = ( Cr * abs(tf - ta) ) / ( mx * abs(tf - t0));
	double ddCr = ( mr * abs(tf - ta) ) / ( mx * abs(tf - t0));
	double ddKter = abs(tf - ta) / ( mx * abs(tf - t0));
	double ddtf =  ( ( (ma * Ca) + (mr * Cr) + Kter ) * (ta - t0) ) / (mx * 
(t0 - tf) * (t0 - tf) );
	double ddta = - ( (ma *Ca) + (mr * Cr) + Kter ) / ( mx * ( tf - t0));
	double ddmx = - ( ( (ma * Ca ) + (mr * Cr) + Kter ) * abs(tf - ta) ) / ( 
mx *  mx * abs(tf -t0) );
	double ddt0 = ( ( (ma * Ca ) + (mr * Cr) + Kter ) * abs(tf - ta) * mx ) 
/ ( mx * mx * (tf - t0) * (tf - t0));
	sCx = sqrt ((ddma * ddma * sma * sma) +
			(ddmr * ddmr * smr * smr) + (ddKter * ddKter * sKter * 
sKter) +
			(ddCr * ddCr * sCr * sCr) + (ddtf * ddtf * stf * stf) +
			(ddta * ddta * sta * sta) + (ddmx * ddmx * smx * smx) + 
(ddt0 * ddt0 * st0 * st0) );
	cout << Cx << " pm " << sCx;
	return 0;
}



Media Pesata
 
//
//  Created by Simone Frau on 22/03/14.
//
//

#include <iostream>
#include <cmath>
#include <fstream>
#include <cstdlib>
#include <string>

using namespace std;

int main ()
{
    int n;
    cout << "dire di quanti valor si vuole calcolare la media" << endl;
    cin >> n;
    double* misure = new double [n];
    for (int i = 0 ; i < n ; i++)
    {
        cout << "inserire valore "<< i << " della media" << endl;
        cin >> misure[i];
    }

    double* sigme = new double [n];
    for (int i = 0 ; i < n ; i++)
    {
        cout << "inserire valore "<< i << " della sigma" << endl;
        cin >> sigme[i];
    }
    
    double sommavalsig;
    for (int i = 0 ; i < n ; i++)
    {
        sommavalsig += (misure[i]/sigme[i]);
    }
    
    double sommak;
    for (int i = 0 ; i < n ; i++)
    {
        sommak += (1/sigme[i]);
    }
    
    double mediapesata;
    mediapesata = (1/sommak)*(sommavalsig);
    
    cout << " Media pesata: " << mediapesata << endl;
    
    double errormediapesata;
    errormediapesata = sqrt(1/sommak);
    
    cout << " Error media pesata: " << errormediapesata << endl;
    
    return 0;
}

Media, Scarto, Sigma

#include <dirent.h>
#include <iostream>
#include <cmath>
#include <fstream>
#include <cstdlib>
#include <string>

using namespace std;

int main ()
{

    string path = "/Users/FrodoFrau/Documents/Università/Fisica Sperimentale 
teoria/Laboratorio/DATI/";
    
    DIR *dir = opendir( path.c_str() );
    if (!dir)
    {
        cerr << "Dir not fnd" << endl;
        exit(1);
    }
    dirent *entry;
    
    while ((entry = readdir(dir)))
    {
        cout << "Found directory entry: "
        << entry->d_name << endl;
    }
    cout << "Di quale file vuoi fare la Media Aritmetica?" << endl;
    
    string filename;
    getline (cin, filename);
    bool flag = 0;
    closedir(dir);
    
    
    path += filename;
    
    int dimensione = 0;
    double valori;
    
    fstream fin ( path.c_str(), fstream::in );

    while (fin >> valori )
    {
        dimensione++;
    }
    fin.close();
    
    fin.open( path.c_str(), fstream::in );
    
    double* misure = new double [dimensione];
    for( int i = 0 ; i < dimensione ; i++ )
    {
        fin>>misure[i];
    }
    
    double sommatoria;
    for (int i = 0 ; i < dimensione ; i++)
    {
        sommatoria +=  misure[i];
    }
    double media = sommatoria/dimensione;
    cout << "Media: " << media << endl;
    
    double differenza;
    for (int i = 0 ; i < dimensione ; i++)
    {
        differenza += (misure[i]-media) * (misure[i]-media);
    }
    double Scarto = sqrt(differenza/(dimensione));
    cout << "Scarto: " << Scarto << endl;
    
    double Sigma = sqrt(differenza/(dimensione-1));
    cout << "Sigma " << Sigma << endl;
    
    double ErroreScarto = Scarto/sqrt(dimensione);
    cout << "ErroreMedia " << ErroreScarto << endl;
    
    return 0;
}


Compatibilità.h

//
//  Created by Simone Frau & Chiappara Davide on 13/01/14.
//
//

#ifndef _Compatibilita__h
#define _Compatibilita__h

#include <dirent.h>
#include <iostream>
#include <cmath>
#include <fstream>
#include <cstdlib>
#include <string>

using namespace std;


void Compatibilità ()
{
    
    double media_a;
    double media_b;
    double sigma_a;
    double sigma_b;
    
    cout << "inserire prima media" << endl;
    cin >> media_a;
    cout << "inserire seconda media" << endl;
    cin >> media_b;
    cout << "inserire primo sigma" << endl;
    cin >> sigma_a;
    cout << "inserire secondo sigma" << endl;
    cin >> sigma_b;
    double Compatibilità = ( abs ( media_a - media_b ) / ( sqrt ( sigma_a * 
sigma_a + sigma_b * sigma_b ) ) );
    
    
    cout << Compatibilità << endl;
    
    
}
            
        
#endif


Interpolazione

#include <iostream>
#include <cmath>

using namespace std;

int main ()
{
	cout << "Si digitino il numero delle coppie di cui fare la 
interpolazione lineare" << endl;
	int dimensione;
	cin >> dimensione;
	double *misurex = new double [dimensione];
	cout << "Si inseriscano le x" << endl;
	for (int i=0 ; i < dimensione ; i++ )
	{
		cin >> misurex[i];
	}
    double *misurey = new double [dimensione];
	cout << "Si inseriscano le y" << endl;
		for (int i=0 ; i < dimensione ; i++ )
	{
		cin >> misurey[i];
	}
	double delta;
	double sommaQUADx = 0;
	for (int i = 0 ; i < dimensione ; i++)
	{
		sommaQUADx += ( misurex[i] * misurex[i] );
	}
	double sommax = 0;
	for ( int i = 0 ; i < dimensione ; i ++)
	{
		sommax += misurex[i];
	}
	double sommaxQUAD = sommax * sommax;
	delta = (dimensione * sommaQUADx) - sommaxQUAD;
	double a,b;
	double sommay = 0;
	for ( int i = 0 ; i < dimensione ; i ++)
	{
		sommay += misurey[i];
	}
	double sommaxy = 0;
	for ( int i = 0 ; i < dimensione ; i ++)
	{
		sommaxy += misurey[i] * misurex[i];
	}
	a = ( 1 / delta ) * ( ( sommaQUADx * sommay ) - ( sommax * sommaxy ) );
	b = ( 1 / delta ) * ( ( ( dimensione * ( sommaxy ) ) - ( sommax * sommay 
) ));
	double sigmay;
	double sommasigma;
	for ( int i = 0 ; i < dimensione ; i++ )
	{
		sommasigma += ( (a + ( b * misurex[i] ) -misurey[i] ) * (a + ( b 
* misurex[i] ) - misurey[i] ) );
	}
	sigmay = sqrt ( sommasigma / ( dimensione -2 ) );
    
    double sigmab = (sqrt ( dimensione/delta )) * sigmay;
    double sigmaa = (sqrt ( sommaQUADx/delta )) * sigmay;

    
	cout << "La a vale: " << a << endl;
	cout << "La b vale: " << b << endl;
	cout << "La sigma vale: " << sigmay << endl;
    cout << "La sigma di b vale " << sigmab << endl;
    cout << "La sigma di a vale " << sigmaa << endl;

                       
	return 0;
}

\end{verbatim}
