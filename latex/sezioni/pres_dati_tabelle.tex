%\begin{tabella}
%	\centering
%	\input{./tabelle/2030d.tex}
%	\caption{Periodo di oscillazione [s] in posizione dritta, intervalli di 2cm}
%	\label{tab:dec}
%\end{tabella}

%\begin{itemize}
%	\item Calore specifico del rame stagnato: $c_r = 0.093 \pm  0.001$, 
%	\item Capacità equivalente termometro: $k_{term} = 2.20 \pm 0.05$,
%	\item Calore specifico acqua: $c_a = 1$
%\end{itemize}

\begin{tabella}
	\centering
	\begin{tabular}{cccccc}
\toprule
 & \multicolumn{2}{c}{Temp. $[\gradi{C}]$} & \multicolumn{3}{c}{Massa $[g]$} \\
 % il primo (argomento) di cmidrule è da che lato eventualmente accorciare la linea, in questo caso le accorcio
 % da entrambi i lati
  \cmidrule(lr){2-3}  \cmidrule(lr){4-6}
 & $T_a$ & $T_0$ & $m_r$ & $m_x$ & $m_a$ \\ \midrule
\emph{Test 1} & 20.2 & 98.6 & 57.92 & 81.73 & 124.06 \\ \midrule
\emph{Test 2} & 20.4 & 98.6 & 57.92 & 81.70 & 62.85 \\ \midrule
\emph{Test 3} & 20.9 & 98.8 & 57.89 & 81.71 & 66.70 \\ \midrule
\emph{Test 4} & 21.3 & 98.8 & 57.88 & 81.72 & 85.01 \\ \midrule
$\sigma$ & 0.3 & 1.5 & 0.02 & 0.01 & 0.03 \\ \bottomrule
\end{tabular}

	\caption{Dati iniziali degli esperimenti}
	\label{tab:ad}
\end{tabella}

\begin{tabella}
	\centering
	\begin{tabular}{cc}
\toprule
\multicolumn{1}{c}{Tempo $[s]$} & \multicolumn{1}{c}{Temperatura $[\gradi C]$} \\ \hline
5 & 21.9 \\ \hline
10 & 22.6 \\ \hline
15 & 24.5 \\ \hline
20 & 25.5 \\ \hline
25 & 28.3 \\ \hline
30 & 28.5 \\ \hline
35 & 28.6 \\ \hline
40 & 28.6 \\ \hline
45 & 28.7 \\ \hline
207 & 28.6 \\ \hline
424 & 28.5 \\ \hline
685 & 28.4 \\ 
\bottomrule
\end{tabular}


	\caption{Primo esperimento}
	\label{tab:t1}
\end{tabella}

\begin{tabella}
	\centering
	\begin{tabular}{cc}
\toprule
\multicolumn{1}{c}{Tempo $[s]$} & \multicolumn{1}{c}{Temperatura $[\gradi C]$} \\ \hline
5 & 22.5 \\ \hline
10 & 24.5 \\ \hline
15 & 25.9 \\ \hline
20 & 26.6 \\ \hline
25 & 26.9 \\ \hline
30 & 27.5 \\ \hline
35 & 27.6 \\ \hline
40 & 27.7 \\ \hline
45 & 27.8 \\ \hline
50 & 27.9 \\ \hline
174 & 27.8 \\ \hline
256 & 27.7 \\ \hline
342 & 27.6 \\ 
\bottomrule
\end{tabular}

	\caption{Secondo esperimento}
	\label{tab:t2}
\end{tabella}

\begin{tabella}
	\centering
	\begin{tabular}{cc}
\toprule
\multicolumn{1}{c}{Tempo $[s]$} & \multicolumn{1}{c}{Temperatura $[\gradi C]$} \\ \hline
5 & 23.5 \\ \hline
10 & 25.4 \\ \hline
15 & 26.5 \\ \hline
20 & 27.6 \\ \hline
25 & 28.2 \\ \hline
30 & 28.6 \\ \hline
35 & 28.9 \\ \hline
40 & 29.1 \\ \hline
45 & 29.3 \\ \hline
50 & 29.5 \\ \hline
55 & 29.6 \\ \hline
60 & 29.7 \\ \hline
65 & 29.7 \\ \hline
70 & 29.8 \\ \hline
75 & 29.8 \\ \hline
80 & 29.8 \\ \hline
85 & 29.9 \\ \hline
187 & 29.8 \\ \hline
243 & 29.7 \\ \hline
302 & 29.6 \\ \hline
361 & 29.5 \\ 
\bottomrule
\end{tabular}

	\caption{Terzo esperimento}
	\label{tab:t3}
\end{tabella}

\begin{tabella}
	\centering
	\begin{tabular}{cc}
\toprule
\multicolumn{1}{c}{Tempo $[s]$ } & \multicolumn{1}{c}{Temperatura $[\gradi C]$} \\ \hline
5 & 23.4 \\ \hline
10 & 25.9 \\ \hline
15 & 27.6 \\ \hline
20 & 28.2 \\ \hline
25 & 28.6 \\ \hline
30 & 28.7 \\ \hline
35 & 28.8 \\ \hline
40 & 28.8 \\ \hline
45 & 28.8 \\ \hline
50 & 28.9 \\ \hline
96 & 28.8 \\ \hline
157 & 28.7 \\ \hline
238 & 28.6 \\ 
\hline
\end{tabular}

	\caption{Quarto esperimento}
	\label{tab:t4}
\end{tabella}

%\[ y = (28.71 \pm 0.01) \grad C + (-0.00046 \pm 0.00003) \grad C / s\]
%\[ y = (57.96 \pm 0.01) \grad C + (-0.00103 \pm 0.00007) \grad C / s\]
%\[ y = (30.04 \pm 0.03) \grad C + (-0.0015 \pm 0.0001) \grad C / s\]
%\[ y = (28.96 \pm 0.02) \grad C + (-0.0015 \pm 0.0001) \grad C / s\]
