Per poter raggiungere gli obiettivi prefissati, si è ripetuto l’esperimento nella sua interezza più volte. L’esperimento è stato suddiviso 
in fasi successive per un’ottimizzazione dei risultati e del tempo impiegato.
\begin{description}

  \item[Fase 1] \hfill \\
	  Misurazione delle masse: per prima cosa si sono misurate sulla bilancia di precisione le masse che venivano inserite all’interno 
	  del calorimetro: il recipiente per l’acqua assieme all’agitatore e il cilindro in acciaio. Successivamente è stato riempito il 
	  contenitore con dell’acqua sufficiente a permettere l’immersione totale del cilindro, ed il sistema recipiente-agitatore-acqua è 
	  stato nuovamente pesato per avere una stima della massa dell’acqua utilizzata.
  \item[Fase 2] \hfill \\
	  Preparazione: per preparare l’esperimento per prima cosa si è messo il cilindro a riscaldare nello strumento apposito, dopodiché 
	  si è posto nel calorimetro il recipiente con l’acqua, si è bloccato il coperchio, avvitato l’agitatore al supporto presente sul 
	  tappo del calorimetro, e si e immerso in acqua il termometro (sempre attraverso l’apposito supporto).
  \item[Fase 3] \hfill \\
	Misurazione dei tempi: non appena il riscaldatore ha portato il cilindro a 
	temperatura (valore che dipendeva, di volta in volta, a causa delle condizioni del riscaldatore e delle condizioni ambientali e fluttuazioni casuali), 
	si è agitata l’acqua grazie all’agitatore, e registrata la misura della temperatura mostrata dal termometro immerso in acqua. 
	Successivamente, il più rapidamente possibile, si è preso il cilindro grazie al filo a cui era attaccato e si è immerso nell’acqua
	posta dentro il calorimetro facendolo passare dall’apposito buco, repentinamente chiuso. Un altro sperimentatore ha cercato con 
	maggior precisione possibile di avviare il cronometro nel momento in cui il cilindro si è immerso in acqua. Per limitare gli errori
	di zero il movimento dello sperimentatore che ha spostato il cilindro in acciaio è stato composto di gesti ampi e quindi 
	prevedibile per il secondo sperimentatore, che ha cercato nel miglior modo possibile di riconoscere, nonostante l’assenza del 
	contatto visivo diretto, il momento in cui il cilindro si immergeva in acqua. A intervalli regolari ($5.0 \pm 0.1 $) $s$ è stata 
	registrata la temperatura mostrata dal cronometro immerso in acqua, fino al momento in cui essa non è risultata stabile nel tempo.
	A quel punto si è misurato il tempo segnato dal cronometro durante il cambio della temperatura mostrata dal termometro (secondo la 
	sua scala di riferimento, quindi ogni $0.1 \gradi C$). Preso un numero di misurazioni ritenuto sufficiente si è interrotto l’esperimento 
	aprendo il calorimetro. L’acqua è stata buttata, tutti i componenti sono stati asciugati e si è ripetuto l’esperimento dalla Fase 
	1.
\end{description}

