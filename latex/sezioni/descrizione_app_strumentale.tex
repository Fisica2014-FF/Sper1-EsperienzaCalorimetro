L’apparato strumentale, atto alla misurazione di tempi e di temperature, era 
formato da:
\begin{itemize}
 \item Calorimetro delle mescolanze: contenitore in grado di ridurre al minimo 
possibile lo scambio di calore tra l’ambiente al suo interno
 e quello al suo esterno: è di forma cilindrica e dotato di tappo ermetico.
 \item Contenitore per acqua: piccolo contenitore di rame stagnato delle 
dimensioni tali da essere posto dentro il calorimetro.
 \item Agitatore: piccola asta in rame stagnato munita di farfalla atta al 
mescolamento dell’acqua all’interno del calorimetro: può essere
    utilizzato inserendolo in un apposito buco posto sul coperchio ermetico del 
calorimetro.
 \item Cilindro in acciaio: Piccolo cilindro di cui si vuole conoscere il 
calore specifico, inseribile all’interno del calorimetro tramite 
    un buco richiudibile presente sul coperchio del calorimetro. Dotato di filo 
termicamente isolato che ne permetta lo spostamento.
 \item Termometro: rilevatore di temperatura a mercurio con sensibilità $10
\gradi C$ e capacità termica fornita di 
 $ (2.20 \pm 0.05) \frac{cal} {\gradi K}$. 
    Può essere inserito parzialmente nel calorimetro attraverso un apposito 
foro presente nel coperchio del calorimetro stesso.
 \item Riscaldatore: elemento in grado di riscaldare un corpo per mezzo di una 
resistenza elettrica. È dotato di un substrato in grado di 
    riscaldare il cilindro in acciaio e di un termometro interno che indica la 
temperatura della resistenza e quella del corpo messo a 
    scaldare con una sensibilità di $10 \gradi C$.
 \item Cronometro digitale: cronometro digitale ad avvio manuale in grado di 
misurare intervalli di tempo, a sensibilità regolabile per 
    mezzo di una manopola. Sensibilità utilizzata per l’esperimento: $ 10 s$.
 \item Bilancia digitale: Bilancia digitale di precisione con sensibilità $10^5 
Kg$
\end{itemize}

