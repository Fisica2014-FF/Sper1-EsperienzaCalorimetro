Erano forniti i seguenti dati:
\begin{itemize}
 \item Calore specifico del rame stagnato: $c_r = (0.093 \pm  0.001) \frac{cal}{g \gradi{K}}$,
 \item Capacità equivalente termometro: $k_{term} = (2.20 \pm 0.05) \frac{cal}{\gradi{K}}$,
 \item Calore specifico acqua: $c_a = 1 \frac{cal}{g \gradi{K}}$
\end{itemize}
%Calore specifico del rame stagnato: $c_r = 0.093 \pm  0.001$,
%Capacità equivalente termometro: $k_{term} = 2.20 \pm 0.05$,
%Calore specifico acqua: $c_a = 1$
%$c_r = (0.093 \pm  0.001) \frac{cal}{g \gradi{K}}$, \\
%$k_{term} = (2.20 \pm 0.05) \frac{cal}{\gradi{K}}$, \\
%$c_a = 1 \frac{cal}{g \gradi{K}}$
Usando i seguenti valori come errori, e propagandoli nella successiva equazione,
\[\sigma _{m_x} = 0.01 g \]
\[\sigma_{m_r+ag} = 0.02 g \]
\begin{equation}
 C_x = \frac{(m_a c_c + m_{r+ag} c_{r+ag} + K_{term}) |T_f - T_{0,a}|}{m_x |T_f - T_x|}
\end{equation}
sono risultati i seguenti valori:
\begin{tabella}
	\centering
	
	\begin{tabular}{ccc}
	    \toprule
	    & $C_x$ & $\sigma$  \\ \midrule
	    \emph{Test 1} & 0.196 & 0.008  \\ \midrule
	    \emph{Test 2} & 0.092 & 0.004  \\ \midrule
	    \emph{Test 3} & 0.120 & 0.005  \\ \midrule
	    \emph{Test 4} & 0.124 & 0.006  \\ 
	    \bottomrule
	\end{tabular}
	
	\caption{Risultati parziali $C_x$}
	\label{tab:ad}
\end{tabella}
da cui poi è stato ottenuto
\begin{equation}
\overline{C}_x = (0.12 \pm 0.04) \frac{cal}{g \gradi{K}}
\end{equation}

Poichè il secondo test è risultato errato, il calcolo è stato rifatto senza di esso ed è stato ottenuto invece questo valore:
\begin{equation}
  \overline{C}_x^{(2)} = 0.14 \pm 0.04 \frac{cal}{g \gradi{K}}
\end{equation}
Si considera più attendibile la seconda media pesata per i motivi di cui seguito discussi.
Ecco le compatibilità dei test:
\[\lambda_{\overline{C}^{(2)}_{x1}} = 1.2 \frac{cal}{g \gradi{K}} \]
\[\lambda_{\overline{C}^{(2)}_{x2}} = 0.4 \frac{cal}{g \gradi{K}} \]
\[\lambda_{\overline{C}^{(2)}_{x3}} = 0.4 \frac{cal}{g \gradi{K}} \]

L'esperimento rivela un valore abbastanza in linea con la teoria. Durante l'esperimento diversi fattori
possono aver influenzato i dati che sono stati raccolti, tra i quali:
\begin{itemize}
\item Influenza dell'ambiente sull'acqua: l'ambiente ha decisamente influenzato le misure che sono state prese, ciò si può
riconoscere in primis dalle diverse temperature ambientali registrate nell'acqua. Dal primo all'
ultimo test la temperatura è variata di $ \approx 1.1 \gradi C$. Questo vuol dire che mediamente la temperatura è 
variata di circa $0.3\gradi C$ tra un campione e l'altro. Per considerare questa influenza si 
è preso tale numero come l'errore sulla sulla temperatura dell'acqua.
\item Influenza dell'ambiente sull'alluminio: il cilindro di alluminio si è raffreddato nel tragitto che porta
 dal riscaldatore che ha misurato la temperatura al calorimetro nel quale esso è stato immerso. Per ovviare
 a tale problema la misura della temperatura è stata abbassata di $1 \gradi C$ e si è considerato un errore di 
$1.5 \gradi C$.
\item Influenza del non perfetto isolamento del calorimetro: Il calorimetro non era perfettamente isolato,
 ma la temperatura tendeva a scendere a causa di dispersioni. Per ovviare a questo problema non è stato preso come
 massimo valore di temperatura raggiunto il massimo valore misurato, ma l'intercetta della retta interpolante
 gli ultimi valori di temperatura misurati (cioè quelli misurati durante il raffreddamento del calorimetro, comunque
non molto diversi), considerando la perdita di calore lineare.
\end{itemize}
Inoltre sono stati presi diversi provvedimenti per la limitazione di errori casuali che potevano essere sottostimati
 a causa del fatto che alcune misure non sono state sufficientemente ripetute. In particolare, l'errore delle masse è
 stato stimato a partire dalle esigue misure ripetute: misurando le masse prima di ogni esperimento, oltre ad
 aver ridotto la probabilità di presenza di eventuali gocce d'acqua residue sui materiali che avrebbero influenzato la
 misurazione successiva, si è permesso il calcolo diretto della sigma su un campione di misure ripetute.
 I valori ottenuti, si rivelano effettivamente buoni come campione
 di stima del calore specifico.
 è stata leggermente problematica la lettura dei valori di temperatura, infatti non è stato semplicissimo per gli sperimentatori
 indagare in quale preciso istante la temperatura cambiava di una tacca di lettura. Comunque si è agito per
 ridurre gli errori di parallasse e di non coordinamento tra gli sperimentatori impegnati.
 La propagazione errori è stata effettuata con l'approssimazione attraverso le derivate parziali al primo ordine.
 Sono stati provati diversi approcci per l'analisi dati: per esempio, si è provato a rappresentare i grafici al
 contrario (cioè scambiando variabile indipendente e dipendente, ossia tempo e temperatura), ma tale analisi rivelava
 dei valori di intercetta i quali errori erano più alti (infatti gli errori andavano propagati, sebbene l'utilizzo
 della stima della covarianza). Inoltre è stato preferito un approccio di stima del calore specifico attraverso
 i vari campioni e poi di media pesata dei valori ottenuti perché non sarebbe stato fisicamente sensato fare una
 media delle grandezze misurate per poi ottenere direttamente un valore unico.
 Il secondo test risulta effettivamente poco compatibile con altri test effettuati, probabilmente a causa di un errore
 ottenuto in sede sperimentale: la troppo bassa quantità d'acqua non ha ricoperto totalmente il cilindro che,
 parzialmente all'aria, ha creato delle condizioni non ottimamente interpretate dalle analisi effettuate,
 ciò è riconoscibile da una temperatura finale decisamente fuori range considerando la quantità d'acqua e gli altri esperimenti 
 effettuati.
 
